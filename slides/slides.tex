\documentclass[aspectratio=169]{beamer}
% remove aspectratio=169 for the standard 4:3 layout

% use the metropolis theme
\usetheme[progressbar=head]{metropolis}
% see metropolis package for more options
% dark theme must be set after the metroepfl package is loaded (see below)

% use the epfl color theme
% defines epfl{gray,darkgray,lightgray,red,blue}
% this usepackage is optional, as it is also done by metroepfl
\usepackage{epflcolors}
% metropolis theme based on gemini and epflcolors
\usepackage{metroepfl}
% \metroset{background=dark}
% multimedia for video/audio
\usepackage{multimedia}
% lilypond (staffsize corrected for beamer document size
\usepackage[staffsize=17]{lyluatex}

\title{Test Slides}
\subtitle{Just for testing the theme}
\author{Max Mustermensch}

\begin{document}

\maketitle

\begin{frame}{First Frame}
  \begin{itemize}
  \item text
  \item \alert{alert text}
  \end{itemize}
  \begin{block}{A Block}
    with some content
  \end{block}
  \begin{exampleblock}{Example}
    This is an example
  \end{exampleblock}
\end{frame}

\begin{frame}[standout]
  Really Important Message!
\end{frame}

\begin{frame}{Audio example}
  The \texttt{multimedia} package is part of beamer and described in the \href{https://ctan.org/pkg/beamer}{\alert{beamer docs}}, Chapter 14.
  Media files are \alert{not} embedded into the PDF and must be kept alongside with the document.

  To include audio (with controls) use the \texttt{movie} command with an audio file (ideally some common format):\\
  \movie[showcontrols,width=10em,height=1em]{\includegraphics[height=1em]{figures/play.pdf}}{example_audio.flac}

  Some PDF viewers (e.g. \href{https://pdfpc.github.io/}{\texttt{pdfpc}}) work better with a video video file,
  otherwise they won't show controls:\\
  \movie[showcontrols,width=10em,height=1em]{\includegraphics[height=1em]{figures/play.pdf}}{example_video.ogv}
\end{frame}

\begin{frame}[fragile]{Lilypond Example}
  You can use \href{https://github.com/jperon/lyluatex}{\texttt{lyluatex}} with LuaTeX
  to typeset lilypond directly in a document.
  This requires lilypond to be installed and doesn't work on Overleaf.

  To use literal lilypond code in beamer, add \texttt{fragile} to the frame.

  \begin{lilypond}
    \relative c' { bes a c b }
  \end{lilypond}

  Including a lilypond file doesn't require the \texttt{fragile} option.

  \lilypondfile{figures/allemande.ly}
\end{frame}

\end{document}
% Local Variables:
% TeX-engine: luatex
% TeX-command-extra-options: "-shell-escape"
% End: